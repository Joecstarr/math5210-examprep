\documentclass[17pt]{extarticle}


\usepackage{amsmath, amssymb, amsthm, amsfonts, mathrsfs}
\usepackage{times, flexisym, mdframed, xcolor}
\usepackage{ulem,multicol}
\usepackage{mathtools}
\usepackage{tikz}
\usepackage{hyperref}
\usepackage{graphicx}
\usepackage{fancyhdr}
\usepackage{tikz-cd}%   Margins
% \usepackage[left=1in,right=1in, top=2in, bottom=2in]{geometry}
\usepackage[paperwidth= 8in,paperheight=5in,left=.25in,right=.25in, top=.25in, bottom=.25in]{geometry}
%

\usepackage{draculatheme}
\mdfdefinestyle{darkAnswer}{%
  fontcolor=draculafg,
  backgroundcolor=draculabg,
  linecolor=draculafg,
  }
\mdfdefinestyle{darkQuesion}{%
  fontcolor=draculafg,
  backgroundcolor=draculabg,
  linecolor=draculafg,
  linewidth=1pt
  }
% \mdfdefinestyle{darkAnswer}{%
%    }
% \mdfdefinestyle{darkQuesion}{%
%   linewidth=1pt
%    }

% --------------------------------------------------------------
%                         New Commands
% --------------------------------------------------------------
\newcommand{\m}{\scalebox{0.5}[1.0]{$-$}}
\newcommand{\lrb}[1]{\left[#1\right]}
\newcommand{\lrp}[1]{\left(#1\right)}
\newcommand{\lrs}[1]{\left\{#1\right\}}
\newcommand{\lra}[1]{\left<#1\right>}
\newcommand{\gof}[1]{g\left(#1\right)}
\newcommand{\fof}[1]{f\left(#1\right)}
\newcommand{\dof}[1]{d\left(#1\right)}
\newcommand{\kof}[1]{k\left(#1\right)}
\newcommand{\mof}[1]{m\left(#1\right)}
\newcommand{\pof}[1]{\phi\left(#1\right)}
\newcommand{\om}[1]{m^{\ast}\left(#1\right)}
\newcommand{\measure}[1]{m\left(#1\right)}
\newcommand{\supmet}[1]{\left\|#1\right\|_{\infty}}
\newcommand{\hof}[1]{h\left(#1\right)}
\newcommand{\met}[3]{\rho_{#1}\lrp{#2,#3}}
\newcommand{\IR}{\mathscr{R}}
\newcommand{\rank}[1]{\text{rank}\left(#1\right)}
\newcommand{\sof}[1]{\sigma\left(#1\right)}
\newcommand{\Zof}[1]{Z\left(#1\right)}
\newcommand{\gofinv}[1]{g^{\m1}\left(#1\right)}
\newcommand{\fofinv}[1]{f^{\m1}\left(#1\right)}
\newcommand{\Zofinv}[1]{Z^{\m1}\left(#1\right)}
\newcommand{\pofinv}[1]{\ph{i=1}^{\m1}\left(#1\right)}
\newcommand{\nmod}[2]{#1\,\left(\text{mod}\,#2\right)}
\newcommand{\ngcd}[2]{\text{gcd}\left(#1\,,\,#2\right)}
\newcommand{\nlcm}[2]{\text{lcm}\left(#1\,,\,#2\right)}
\newcommand{\grp}[2]{\left(#1\,,\,#2\right)}
\newcommand{\GL}[2]{\mathrm{GL}_{#1}\lrp{#2}}
\newcommand{\SL}[2]{\mathrm{SL}_{#1}\lrp{#2}}
\newcommand{\Mn}[2]{M_{#1}\left(#2\right)}
\newcommand{\nord}[2]{\text{ord}_{#1}\left(#2\right)}
\newcommand{\ord}[1]{\text{ord}\left(#1\right)}
\newcommand{\dom}[1]{\text{dom}\left(#1\right)}
\newcommand{\ran}[1]{\text{ran}\left(#1\right)}
\newcommand{\degr}[1]{\text{deg}\left(#1\right)}
\newcommand{\krn}[1]{\text{ker}\left(#1\right)}
\newcommand{\intr}[1]{\text{int}\left(#1\right)}
\newcommand{\ball}[2]{B_{#1}\left(#2\right)}
\newcommand{\N}{\mathbb{N}}
\newcommand{\Z}{\mathbb{Z}}
\newcommand{\Q}{\mathbb{Q}}
\newcommand{\R}{\mathbb{R}}
\newcommand{\Opn}{\mathcal{O}}
\newcommand{\F}{\mathcal{F}}
\newcommand{\notsubset}{\not\subset}
% \newcommand{\right)}{\mathbb{R}^{+}}
\newcommand{\Dn}{\Delta^{n}}
\renewcommand{\C}{\mathbb{C}}
\newcommand{\primeDec}{p_1^{\alpha_1}p_2^{\alpha_2}\cdots p_m^{\alpha_m}p_{m+1}^{\alpha_{m+1}}}
\newcommand{\abs}[1]{\left\vert#1\right\vert}
\newcommand{\norm}[1]{\left\vert\left\vert#1\right\vert\right\vert}
\newcommand{\Zm}[1]{\mathbb{Z}_{#1}}
\newcommand{\Zmx}[1]{\mathbb{Z}_{#1}^{\times}}
\newcommand{\Zp}{\mathbb{Z}_p}
\newcommand{\numb}[1]{\noindent{\bf #1)}}
\newcommand{\bigslant}[2]{{\raisebox{.2em}{$#1$}\left/\raisebox{-.2em}{$#2$}\right.}}
\newcommand{\inv}[1]{#1^{\m1}}
\newcommand{\totient}[1]{\varphi\left(#1\right)}
\newcommand{\vhalfpg}{\vspace{5in}}
\newcommand{\vthirdpg}{\vspace{3in}}
\newcommand{\vquartpg}{\vspace{2in}}
\newcommand{\Assoc}[1]{\item[Associativity:]{#1}}
\newcommand{\Invs}[1]{\item[Inverses:]{#1}}
\newcommand{\Clos}[1]{\item[Closure:]{#1}}
\newcommand{\Ident}[1]{\item[Identity:]{#1}}
\newcommand{\Abel}[1]{\item[Abelian:]{#1}}
\newcommand{\Tv}{\text{Tv}}
\newcommand{\V}{\text{V}}
\newcommand{\Ip}[1]{\text{Im}#1}
\newcommand{\LP}{\left(}
\newcommand{\RP}{\right)}
\newcommand{\LS}{\left\lbrace}
\newcommand{\RS}{\right\rbrace}
\newcommand{\LB}{\left[}
\newcommand{\RB}{\right]}
\newcommand{\MM}{\ \middle|\ }
\newcommand{\msr}[1]{m\left(#1\right)}
\newcommand{\dist}[1]{\text{d}\left(#1\right)}
\newcommand{\Diff}[3]{Diff_{#1}#2\left(#3\right)}
\newcommand{\Av}[3]{Av_{#1}#2\left(#3\right)}
\newcommand{\cball}[2]{\overline{B}_{#1}\left(#2\right)}
\newcommand{\opn}{\mathcal{O}}
\newcommand{\diam}{\operatorname{diam}}
\newcommand{\wind}[1]{n\lrp{\gamma;\ #1}}

\DeclarePairedDelimiter\ceil{\lceil}{\rceil}
\DeclarePairedDelimiter\floor{\lfloor}{\rfloor}
\newcommand{\twocase}[2]{\begin{enumerate}
                        \item[$ \implies$]{
                            #1
                            }
                        \bigskip
                        \item[$ \impliedby$]{
                          #2
                          }
                        \end{enumerate}}

% --------------------------------------------------------------
%                         Renew Commands
% --------------------------------------------------------------
% \renewcommand{\det}[1]{\text{det}\left(#1\right)}
\renewcommand{\bar}[1]{\overline{#1}}
% \renewcommand{\cos}[1]{\text{cos}\left(#1\right)}

\newcommand{\boxset}[2]{\begin{mdframed}[style=darkQuesion]
#1
\end{mdframed}
\newpage
\begin{mdframed}[style=darkQuesion]
  #1
    \end{mdframed}
\begin{mdframed}[style=darkAnswer]
  #2
    \end{mdframed}
    \newpage
}

\begin{document}

% --------------------------------------------------------------
%                         Start here
% --------------------------------------------------------------
% \pagestyle{fancy}
% \fancyhf{}
% \rhead{Math 5210 Homework 7}
% \lhead{ }
% \rfoot{Page \thepage}
\centering{{\fontsize{26pt}
{24pt}\selectfont{\underline{\smash{By Heart}}}}}\par
\newpage

\boxset{The Polar representation of complex numbers }
{Given a point $z=x+yi$ in the complex plane. The point has a polar representation $\left( r,\ \theta\right):\ x=r\cos{\theta},\ y=r\sin{\theta}$, where $r=\left|z\right|$ and $\theta$ is the angle between the positive real axis and the line segment from $0$ to $z$.}
\boxset{Roots of complex numbers }
{Given a complex number $a=\left|a\right|\text{cis}\left(\alpha\right)\neq 0$ and an integer $n\geq 2$, a $n^{\text{th}}$ root of $a$ is a number \[\left|a\right|^{\frac{1}{n}}\text{cis}\left( \frac{1}{n}\left( \alpha+2\pi k\right)\right)\] where $0\leq k \leq n-1$ }
\boxset{Lines in $\C$ }
{A line in $\C$ is of the form \[L=\LS z=a+tb\MM -\infty<t<\infty\RS \] for $a,b\in\C$ or \[L=\left\{z: \operatorname{Im}\left(\frac{z-a}{b}\right)=0\right\}\] }
\boxset{Half planes in $\C$ }
{For $a,b\in\C$W e are "walking along $L$ in the direction of $b$." If we put \[H_{a}=\left\{z: \operatorname{Im}\left(\frac{z-a}{b}\right)>0\right\}\] then it is easy to see that $H_{a}=a+H_{0} \equiv\left\{a+w: w \in H_{0}\right\}$; that is, $H_{a}$ is the translation of $H_{0}$ by $a$. Hence, $H_{a}$ is the half plane lying to the left of $L$. Similarly, \[K_{a}=\left\{z: \operatorname{Im}\left(\frac{z-a}{b}\right)<0\right\}\] is the half plane on the right of $L$.}
\boxset{The triangle inequality in $\C$ }
{\[|z+w| \leq|z|+|w|,\;(z, w \in \C)\] where \[|z|=\left(x^{2}+y^{2}\right)^{\frac{1}{2}}\] }
\boxset{The Weierstrass $M$-test for series of functions }
{ Let $u_{n}: X \rightarrow \C$ be a function such that $\left|u_{n}(x)\right| \leq M_{n}$ for every $x$ in $X$ and suppose the constants satisfy $\sum_{n=1}^{\infty} M_{n}<\infty$. Then $\sum_{1}^{\infty} u_{n}$ is uniformly convergent. }
\boxset{The Heine-Borel Theorem }
{A subset $K$ of $\R^n$, $n\geq 1$ is compact iff $K$ is closed and bounded.}
\boxset{The Cantor Intersection Theorem }
{Let $X$ be a metric space. Then $X$ is complete if and only if whenever $\left\{F_{n}\right\}_{n=1}^{\infty}$ is a contracting sequence of nonempty closed subsets of $X$, there is a point $x \in X$ for which $\bigcap_{n=1}^{\infty} F_{n}=\{x\}$}
\boxset{The Cauchy Convergence Criterion }
{If $(X, d)$ has the property that each Cauchy sequence has a limit in $X$ then $(X, d)$ is complete.}
\boxset{The Intermediate Value Theorem }
{If $f:[a, b] \rightarrow \mathbb{R}$ is continuous and $f(a) \leq \xi$ $\leq f(b)$ then there is a point $x, a \leq x \leq b$, with $f(x)=\xi$.}
\boxset{Morera's Theorem }
{ Let $G$ be a region and let $f: G \rightarrow \C$ be a continuous function such that $\int_{T} f=0$ for every triangular path $T$ in $G$; then $f$ is analytic in $G$.}
\boxset{Cauchy's Theorem (Second Version) }
{If $f: G \rightarrow \C$ is an analytic function and $\gamma$ is a closed rectifiable curve in $G$ such that $\gamma \sim 0$, then \[\int_{\gamma} f=0 \text {. }\]}
\boxset{Cauchy's Theorem (Fourth Version) }
{If $G$ is simply connected then  $\int_\gamma f=0$ for every closed rectifiable curve and every analytic function $f$.}
\boxset{Open Mapping Theorem }
{Let $G$ be a region and suppose that $f$ is a non constant analytic function on $G$. Then for any open set $U$ in $G, f(U)$ is open.}
\boxset{Goursat's Theorem }
{Let $G$ be an open set and let $f: G \rightarrow \C$ be a differentiable function; then $f$ is analytic on $G$.}
\boxset{Laurent series development of an analytic function in an annulus }
{Let $f$ be analytic in the annulus ann $\left(a ; R_{1}\right.$, $R_{2}$ ). Then \[f(z)=\sum_{n=-\infty}^{\infty} a_{n}(z-a)^{n}\] where the convergence is absolute and uniform over ann $\left(a ; r_{1}, r_{2}\right)^{-}$if $R_{1}<$ $r_{1}<r_{2}<R_{2}$. Also the coefficients $a_{n}$ are given by the formula \[a_{n}=\frac{1}{2 \pi i} \int_{\gamma} \frac{f(z)}{(z-a)^{n+1}} d z\] where $\gamma$ is the circle $|z-a|=r$ for any $r, R_{1}<r<R_{2}$. Moreover, this series is unique. }
\boxset{Residue Theorem }
{Let $f$ be analytic in the region $G$ except for the isolated singularities \[\ \] $a_{1}, a_{2}, \ldots, a_{m}$. If $\gamma$ is a closed rectifiable curve in $G$ which does not pass through any of the points $a_{k}$ and if $\gamma \approx 0$ in $G$ then \[\frac{1}{2 \pi i} \int_{\gamma} f=\sum_{k=1}^{m} n\left(\gamma ; a_{k}\right) \operatorname{Res}\left(f ; a_{k}\right) \text {. }\]}

\boxset{The Argument Principle }
{Let $f$ be meromorphic in $G$ with poles $p_{1}, p_{2}, \ldots, p_{m}$ and zeros $z_{1}, z_{2}, \ldots, z_{n}$ counted according to multiplicity. If $\gamma$ is a closed rectifiable curve in $G$ with $\gamma \approx 0$ and not passing through $p_{1}, \ldots, p_{m}$; $z_{1}, \ldots, z_{n}$; then \[\frac{1}{2 \pi i} \int_{\gamma} \frac{f^{\prime}(z)}{f(z)} d z=\sum_{k=1}^{n} n\left(\gamma ; z_{k}\right)-\sum_{j=1}^{m} n\left(\gamma ; p_{j}\right) .\] }
\boxset{Rouché's Theorem }
{ Suppose $f$ and $g$ are meromorphic in a neighborhood of $\bar{B}(a ; R)$ with no zeros or poles on the circle $\gamma=\{z:|z-a|=R\} .$ If $Z_{f}, Z_{g}$ $\left(P_{f}, P_{g}\right)$ are the number of zeros (poles) of $f$ and $g$ inside $\gamma$ counted according to their multiplicities and if \[|f(z)+g(z)|<|f(z)|+|g(z)|\] on $\gamma$, then \[Z_{f}-P_{f}=Z_{g}-P_{g} .\]}
\boxset{Schwarz's Lemma}
{Let $D=\{z:|z|<1\}$ and Suppose $f$ is analytic on $D$ with\[\ \](a) $|f(z)| \leq 1$ for $z$ in $D$,\[\ \](b) $f(0)=0$.\[\ \]Then $\left|f^{\prime}(0)\right| \leq 1$ and $|f(z)| \leq|z|$ for all $z$ in the disk D. Moreover if $\left|f^{\prime}(0)\right|=1$ or if $|f(z)|=|z|$ for some $z \neq 0$ then there is a constant $c,|c|=1$, such that $f(w)=c w$ for all $w$ in $D$.}
\boxset{The Arzelà-Ascoli Theorem}
{A set $\mathscr{F} \subset C(G, \Omega)$ is normal iff the following two conditions are satisfied:\[\ \](a) for each $z$ in $G,\{f(z): f \in \mathscr{F}\}$ has compact closure in $\Omega$;\[\ \](b) $\mathscr{F}$ is equicontinuous at each point of $G$.}
\boxset{Hurwitz's Theorem}
{Let $G$ be a region and suppose the sequence $\left\{f_{n}\right\}$ in $H(G)$ converges to $f .$ If $f \not \equiv 0, \bar{B}(a ; R) \subset G$, and $f(z) \neq 0$ for $|z-a|=R$ then there is an integer $N$ such that for $n \geq N, f$ and $f_{n}$ have the same number of zeros in $B(a ; R)$.}
\boxset{Montel's Theorem}
{A family $\mathscr{F}$ in $H(G)$ is normal iff $\mathscr{F}$ is locally bounded.}
\boxset{The Riemann Mapping Theorem}
{Let $G$ be a simply connected region which is not the whole plane and let $a \in G$. Then there is a unique analytic function $f: G \rightarrow \mathbb{C}$ having the properties:\[\ \](a) $f(a)=0$ and $f^{\prime}(a)>0$;\[\ \](b) $f$ is one-one;\[\ \](c) $f(G)=\{z:|z|<1\}$.}
\boxset{Gauss's Formula}
{For $z \neq 0,-1, \ldots$ \[\Gamma(z)=\lim _{n \rightarrow \infty} \frac{n ! n^{z}}{z(z+1) \ldots(z+n)}\]}
\boxset{Functional Equation for $\Gamma$}
{For $z \neq 0,-1, \ldots$ \[\Gamma(z+1)=z \Gamma(z)\]}
\boxset{Mean Value Theorem for harmonic functions}
{If $u: G \rightarrow \mathbb{R}$ is a harmonic function and $\bar{B}(a ; r)$ is a closed disk contained in $G$, then\[u(a)=\frac{1}{2 \pi} \int_{0}^{2 \pi} u\left(a+r e^{i \theta}\right) d \theta\]}
\boxset{Maximum Principle (First Version) for harmonic functions}
{Let $G$ be a region and suppose that $u$ is a continuous real valued function on $G$ with the MVP. If there is a point a in $G$ such that $u(a) \geq u(z)$ for all $z$ in $G$ then $u$ is a constant function.}
\boxset{Minimum Principle for harmonic functions}
{Let $G$ be a region and suppose that $u$ is a continuous real valued function on $G$ with the MVP. If there is a point a in $G$ such that $u(a) \leq u(z)$ for all $z$ in $G$ then $u$ is a constant function.}
\boxset{Harnack's Theorem}
{Let $G$ be a region. (a) The metric space Har( $G)$ is complete. (b) If $\left\{u_{n}\right\}$ is a sequence in $\operatorname{Har}(G)$ such that $u_{1} \leq u_{2} \leq \ldots$ then either $u_{n}(z) \rightarrow \infty$ uniformly on compact subsets of $G$ or $\left\{u_{n}\right\}$ converges in Har $(G)$ to a harmonic function.}
\centering{{\fontsize{26pt }
{24pt}\selectfont{\underline{\smash{Define}}}}}\par \newpage
\boxset{Connectedness }
{ A metric space $(X, d)$ is connected if the only subsets of $X$ which are both open and closed are $\square$ and $X$. If $A \subset X$ then $A$ is a connected subset of $X$ if the metric space $(A, d)$ is connected.}
\boxset{Cauchy sequence }
{ A sequence $\left\{x_{n}\right\}$ is called a Cauchy sequence if for every $\epsilon>0$ there is an integer $N$ such that $d\left(x_{n}, x_{m}\right)<\epsilon$ for all $n, m \geq N$.}
\boxset{Uniform convergence }
{Let $X$ be a set and $(\Omega, \rho)$ a metric space and suppose $f, f_{1}, f_{2}, \ldots$ are functions from $X$ into $\Omega$. The sequence $\left\{f_{n}\right\}$ converges uniformly to $f$-written $f=u-\lim f_{n}$-if for every $\epsilon>0$ there is an integer $N$ (depending on $\epsilon$ alone) such that $\rho\left(f(x), f_{n}(x)\right)<\epsilon$ for all $x$ in $X$, whenever $n \geq N$.}
\boxset{Analytic function }
{ A function $f: G \rightarrow \C$ is analytic if $f$ is continuously differentiable on $G$.}
\boxset{Principal branch of the logarithm }
{ If $G$ is an open connected set in $\C$ and $f: G \rightarrow \C$ is a continuous function such that $z=\exp f(z)$ for all $z$ in $G$ then $f$ is a branch of the logarithm. We designate the particular branch of the logarithm defined above on $\C-\{z: z \leq 0\}$ to be the principal branch of the logarithm.}
\boxset{Definition of Möbius map }
{ A mapping of the form $S(z)=\frac{a z+b}{c z+d}$ is called a linear fractional transformation. If $a, b, c$, and $d$ also satisfy $a d-b c \neq 0$ then $S(z)$ is called a Möbius transformation.}
\boxset{Symmetry Principle }
{ If a Möbius transformation $T$ takes a circle $\Gamma_{1}$ onto the circle $\Gamma_{2}$ then any pair of points symmetric with respect to $\Gamma_{1}$ are mapped by $T$ onto a pair of points symmetric with respect to $\Gamma_{2}$.}
\boxset{Orientation Principle }
{ Let $\Gamma_{1}$ and $\Gamma_{2}$ be two circles in $\C_{\infty}$ and let $T$ be a Möbius transformation such that $T\left(\Gamma_{1}\right)=\Gamma_{2}$. Let $\left(z_{1}, z_{2}, z_{3}\right)$ be an orientation for $\Gamma_{1}$. Then $T$ takes the right side and the left side of $\Gamma_{1}$ onto the right side and left side of $\Gamma_{2}$ with respect to the orientation $\left(T z_{1}, T z_{2}, T z_{3}\right)$.}
\boxset{Riemann-Stieltjes integral }
{ Let $\gamma:[a, b] \rightarrow \C$ be of bounded variation and suppose that $f:[a, b] \rightarrow \C$ is continuous. Then there is a complex number I such that for every $\epsilon>0$ there is a $\delta>0$ such that when $P=\left\{t_{0}<t_{1}<\ldots<t_{m}\right\}$ is a partition of $[a, b]$ with $\|P\|=\max \left\{\left(t_{k}-t_{k-1}\right): 1 \leq k \leq m\right\}<\delta$ then \[\abs{I-\sum_{k=1}^m f\LP\tau_k\RP\LB\gamma\LP t_k\RP-\LP t_{k-1}\RP\RB}<\varepsilon\] for whatever choice of points $\tau_{k}, t_{k-1} \leq \tau_{k} \leq t_{k}$. This number $I$ is called the integral of $f$ with respect to $\gamma$ over $[a, b]$ and is designated by \[I=\int_{a}^{b} f d \gamma=\int_{a}^{b} f(t) d \gamma(t) .\] }
\boxset{Cauchy's Estimate }
{ Let $f$ be analytic in $B(a ; R)$ and suppose $|f(z)| \leq M$ for all $z$ in $B(a ; R)$. Then \[\left|f^{(n)}(a)\right| \leq \frac{n ! M}{R^{n}}\] }
\boxset{Liouville's Theorem }
{ If $f$ is a bounded entire function then $f$ is constant.}
\boxset{Fundamental Theorem of Algebra }
{ If $p(z)$ is a non constant polynomial then there is a complex number $a$ with $p(a)=0$.}
\boxset{Maximum Modulus Theorem }
{ If $G$ is a region and $f: G \rightarrow \C$ is an analytic function such that there is a point $a$ in $G$ with $|f(a)| \geq|f(z)|$ for all $z$ in $G$, then $f$ is constant.}
\boxset{Index of a closed rectifiable curve $\gamma$ in $\C$ with respect to a point $a \notin \gamma$ }
{ If $\gamma$ is a closed rectifiable curve in $\C$ then for $a \notin\{\gamma\}$ \[n(\gamma ; a)=\frac{1}{2 \pi i} \int_{\gamma}(z-a)^{-1} d z\]  is called the index of $\gamma$ with respect to the point $a$. It is also sometimes called the winding number of $\gamma$ around $a$.}
\boxset{When is an open set simply connected? }
{ An open set $G$ is simply connected if $G$ is connected and every closed curve in $G$ is homotopic to zero.}
\boxset{A rectifiable curve homologous to zero }
{ If $G$ is an open set then $\gamma$ is homologous to zero, in symbols $\gamma \approx 0$, if $n(\gamma ; w)=0$ for all $w$ in $\C-G$.}
\boxset{Removable singularity of an analytic function at a point $z=a$ }
{ A function $f$ has an isolated singularity at $z=a$ if there is an $R>0$ such that $f$ is defined and analytic in $B(a ; R)-\{a\}$ but not in $B(a ; R)$. The point $a$ is called a removable singularity if there is an analytic function $g: B(a ; R) \rightarrow \C$ such that $g(z)=f(z)$ for $0<|z-a|<R$.}
\boxset{Pole of a function }
{If $z=a$ is an isolated singularity of $f$ then $a$ is a pole of $f$ if $\lim _{z \rightarrow a}|f(z)|=\infty$. That is, for any $M>0$ there is a number $\epsilon>0$ such that $|f(z)| \geq M$ whenever $0<|z-a|<\epsilon$. If an isolated singularity is neither a pole nor a removable singularity it is called an essential singularity.}
\boxset{Cauchy's Integral Formula (First Version) }
{Let $G$ be an open subset of the plane and $f: G \rightarrow \mathbb{C}$ an analytic function. If $\gamma$ is a closed rectifiable curve in $G$ such that $n(\gamma ; w)=0$ for all $w$ in $\mathbb{C}-G$, then for $a$ in $G-\{\gamma\}$ \[n(\gamma ; a) f(a)=\frac{1}{2 \pi i} \int_{\gamma} \frac{f(z)}{z-a} d z .\]}
\boxset{Cauchy's Integral Formula (Second Version) }
{Let $G$ be an open subset of the plane and $f: G \rightarrow \mathbb{C}$ an analytic function. If $\gamma_{1}, \ldots, \gamma_{m}$ are closed rectifiable curves in $G$ such that $n\left(\gamma_{1} ; w\right)+\cdots+n\left(\gamma_{m} ; w\right)=0$ for all $w$ in $\mathbb{C}-G$, then for a in $G-\cup_{k=1}^{m}\left\{\gamma_{k}\right\}$ \[f(a) \sum_{k=1}^{m} n\left(\gamma_{k} ; a\right)=\sum_{k=1}^{m} \frac{1}{2 \pi i} \int_{\gamma_{k}} \frac{f(z)}{z-a} d z .\]}
\boxset{ If $f$ is analytic on an open connected set $G$ and $f$ is $\rule{1cm}{0.15mm}$, then for each $a$ in $G$ with $\rule{1cm}{0.15mm}$ there is an integer $n \geq 1$ and an $\rule{1cm}{0.15mm}$ $g: G \rightarrow \mathbb{C}$ such that $\rule{1cm}{0.15mm}$ and for all $z$ in $G$. That is, $\rule{1cm}{0.15mm}$  }
{ If $f$ is analytic on an open connected set $G$ and $f$ is not identically zero, then for each a in $G$ with $f(a)=0$ there is an integer $n \geq 1$ and an analytic function $g: G \rightarrow \mathbb{C}$ such that $g(a) \neq 0$ and \[f(z)=(z-a)^{n} g(z)\] for all $z$ in $G$. That is, each zero of $f$ has finite multiplicity.}
\boxset{ If $\gamma:[0,1] \rightarrow \mathbb{C}$ is a closed rectifiable curve and $a \notin\{\gamma\}$ then \[\rule{1cm}{0.15mm}\] is an $\rule{1cm}{0.15mm}$.}{ If $\gamma:[0,1] \rightarrow \mathbb{C}$ is a closed rectifiable curve and $a \notin\{\gamma\}$ then \[\frac{1}{2 \pi i} \int_{\gamma} \frac{d z}{z-a}\] is an integer.}
\boxset{ If $\gamma:[0,1] \rightarrow \mathbb{C}$ is a closed rectifiable curve and $a \notin\{\gamma\}$ then \[\rule{1cm}{0.15mm}\] is an $\rule{1cm}{0.15mm}$.}{ If $\gamma:[0,1] \rightarrow \mathbb{C}$ is a closed rectifiable curve and $a \notin\{\gamma\}$ then \[\frac{1}{2 \pi i} \int_{\gamma} \frac{d z}{z-a}\] is an integer.}
\boxset{  Let $\gamma$ be a closed rectifiable curve in $\mathbb{C}$. Then\[\ \] (i) $n(\gamma ; a)$ $\rule{1cm}{0.15mm}$ of $G=\mathbb{C}-\{\gamma\}$; and\[\ \] (ii) $n(\gamma ; a)=0$ for $a$ belonging $\rule{1cm}{0.15mm}$ $G$. }
{ Let $\gamma$ be a closed rectifiable curve in $\mathbb{C}$. Then\[\ \] (i) $n(\gamma ; a)$ is constant for $a$ belonging to a component of $G=\mathbb{C}-\{\gamma\}$; and\[\ \] (ii) $n(\gamma ; a)=0$ for $a$ belonging to the unbounded component of $G$. is an integer.}
\boxset{Cauchy's Theorem for functions analytic in a disk }
{if $G$ is an open disk then \[\int_{\gamma} f=0\] for any analytic function $f$ on $G$ and any closed rectifiable curve $\gamma$ in $G$.}
\boxset{Cauchy's Integral Formula for Derivatives }
{Let $G$ be an open subset of the plane and $f: G \rightarrow \mathbb{C}$ be an analytic function. Let $\gamma_{1}, \ldots, \gamma_{m}$ be closed rectifiable curves in $G$ such that \[n\left(\gamma_{1} ; w\right)+\cdots+n\left(\gamma_{m} ; w\right)=0\] for all $w$ in $\mathbb{C}-G$. Then for $a$ in $G-\{\gamma\}$ and $k \geq 1$, we have \[f^{(k)}(a) \sum_{j=1}^{m} n\left(\gamma_{j} ; a\right)=k ! \sum_{j=1}^{m} \frac{1}{2 \pi i} \int_{\gamma,} \frac{f(z)}{(z-a)^{k+1}} d z\]}
\boxset{   A region $G$ is $\rule{1cm}{0.15mm}$ if and only if $\mathbb{C}_{\infty}-G$, its complement in the extended plane, is connected in $\mathbb{C}_{\infty}$.  }
{A region $G$ is simply connected if and only if $\mathbb{C}_{\infty}-G$, its complement in the extended plane, is connected in $\mathbb{C}_{\infty}$.}
\boxset{   Let $G$ be $\rule{1cm}{0.15mm}$ and let $f: G \rightarrow \mathbb{C}$ be an analytic function such the $f(z) \neq 0$ for any $z$ in $G$. Then there exists an analytic function $g: G \rightarrow \mathbb{C}$ such that $\rule{1cm}{0.15mm}$. If $z_{0} \in G$ and $e^{w_{0}}=f\left(z_{0}\right)$, we may choose $g$ such that $\rule{1cm}{0.15mm}$  }
{   Let $G$ be simply connected and let $f: G \rightarrow \mathbb{C}$ be an analytic function such the $f(z) \neq 0$ for any $z$ in $G$. Then there exists an analytic function $g: G \rightarrow \mathbb{C}$ such that $f(z)=\exp g(z)$. If $z_{0} \in G$ and $e^{w_{0}}=f\left(z_{0}\right)$, we may choose $g$ such that $g\left(z_{0}\right)=w_{0}$ }
\boxset{Let $z=a$ be an isolated singularity of $f$ and let $f(z)=\sum_{-\infty}^{\infty} a_{n}(z-a)^{n}$ be its Laurent Expansion in ann $(a ; 0, R)$. Then:\[\ \] (a) $z=a$ is a $\rule{1cm}{0.15mm}$ if and only if $a_{n}=0$ for $n \leq-1$;\[\ \] (b) $z=a$ is a $\rule{1cm}{0.15mm}$ of order $m$ if and only if $a_{-m} \neq 0$ and $a_{n}=0$ for $n \leq-(m+1)$;\[\ \] (c) $z=a$ is an$\rule{1cm}{0.15mm}$ if and only if $a_{n} \neq 0$ for infinitely many negative integers $n$. }
{Let $z=a$ be an isolated singularity of $f$ and let $f(z)=\sum_{-\infty}^{\infty} a_{n}(z-a)^{n}$ be its Laurent Expansion in ann $(a ; 0, R)$. Then:\[\ \] (a) $z=a$ is a removable singularity if and only if $a_{n}=0$ for $n \leq-1$;\[\ \] (b) $z=a$ is a pole of order $m$ if and only if $a_{-m} \neq 0$ and $a_{n}=0$ for $n \leq-(m+1)$;\[\ \] (c) $z=a$ is an essential singularity if and only if $a_{n} \neq 0$ for infinitely many negative integers $n$.}
\boxset{Suppose $f$ has a pole of order $m$ at $z=a$ and put $g(z)=\rule{1cm}{0.15mm}$; then \[\rule{1cm}{0.15mm}\] }
{   Suppose $f$ has a pole of order $m$ at $z=a$ and put $g(z)=(z-a)^{m} f(z)$; then \[\operatorname{Res}(f ; a)=\frac{1}{(m-1) !} g^{(m-1)}(a)\] }
\boxset{Polynomially convex hull of a compact set}
{Let $K$ be a compact subset of the plane; the polynomially convex hull of $K$, denoted by $\hat{K}$, is defined to be the set of all points $w$ such that for every polynomial $p$\[|p(w)| \leq \max \{|p(z)|: z \in K\} .\] That is, if the right hand side of this inequality is denoted by $\|p\|_{K}$, then \[\hat{K}=\left\{w:|p(w)| \leq\|p\|_{K} \text { for all polynomials } p\right\} .\]}
\boxset{harmonic conjugate}
{If $f: G \rightarrow \mathbb{C}$ is an analytic function then $u=\operatorname{Re} f$ and $v=\operatorname{Im} f$ are called harmonic conjugates.}
\boxset{Harmonic function}
{If $G$ is an open subset of $\mathbb{C}$ then a function $u: G \rightarrow \mathbb{R}$ is harmonic if $u$ has continuous second partial derivatives and\[\frac{\partial^{2} u}{\partial x^{2}}+\frac{\partial^{2} u}{\partial y^{2}}=0\]}
\boxset{$C(G, \Omega)$}
{If $G$ is an open set in $\mathbb{C}$ and $(\Omega, d)$ is a complete metric space then designate by $C(G, \Omega)$ the set of all continuous functions from $G$ to $\Omega$.}
\boxset{equicontinuous}
{A set $\mathscr{F} \subset C(G, \Omega)$ is equicontinuous at a point $z_{0}$ in $G$ iff for every $\epsilon>0$ there is a $\delta>0$ such that for $\left|z-z_{0}\right|<\delta$,\[d\left(f(z), f\left(z_{0}\right)\right)<\epsilon\]for every $f$ in $\mathscr{F} . \mathscr{F}$ is equicontinuous over a set $E \subset G$ if for every $\epsilon>0$ there is a $\delta>0$ such that for $z$ and $z^{\prime}$ in $E$ and $\left|z-z^{\prime}\right|<\delta$,\[d\left(f(z), f\left(z^{\prime}\right)\right)<\epsilon\]}
\boxset{normal}
{A set $\mathscr{F} \subset C(G, \Omega)$ is normal if each sequence in $\mathscr{F}$ has a subsequence which converges to a function $f$ in $C(G, \Omega)$.}
\boxset{Series representation for $e^z$}
{\[\sum_{k=0}^{\infty} \frac{z^{k}}{k !}\]}
\boxset{Series representation for $\log(z)$}
{\[\sum_{k=1}^{\infty}(-1)^{k+1} \frac{(z-1)^{k}}{k}\]}
\boxset{Series representation for $\sin(z)$}
{\[\sum_{n=0}^{\infty} \frac{(-1)^{n}}{(2 n+1) !} z^{2 n+1}\]}
\boxset{Series representation for $\cos(z)$}
{\[\sum_{n=0}^{\infty} \frac{(-1)^{n}}{(2 n) !} z^{2 n}\]}
\end{document}
